\documentclass[12pt]{article}
\usepackage{amsmath}
\usepackage{amssymb}
\usepackage{graphicx}
\usepackage{hyperref}
\usepackage[latin1]{inputenc}

\title{Squeeze Theorem Lazy Proof}
\author{Jeremy Chow}
\date{10/9/18}

\begin{document}
\maketitle
The squeeze theorem stated that $a_{n}\leq b_{n} \leq c_{n}$, $(a_{n})\to k  $ and $(c_{n})\to k$, then $(b_{n})\to k$. In order to prove this theorem, we need to show that for any $\varepsilon>0$, there exists $N\in \mathbb{N}$ such that $n>N$ implies $|b_{n}-k|<\varepsilon$. Given by the situation, $|a_{n}-k|<\varepsilon$ for $N_{1}$ and $|c_{n}-k|<\varepsilon$ for $N_{2}$, then let $n>\max\{N_{1},N_{2}\}$ so both statements holds true. The lowest that $a_{n}$ will be is given by $k-a_{n}<\varepsilon$, $k-\varepsilon<a_{n}$. The highest that $c_{n}$ will be is given by $c_{n}-k<\varepsilon$, $c_{n}<k+\varepsilon$. So we have $k-\varepsilon<a_{n}\leq b_{n} \leq c_{n}<k+\varepsilon$, $k-\varepsilon<b_{n}<k+\varepsilon$, which can be rewritten as $|b_{n}-k|<\varepsilon$. Hence, by the $N-\varepsilon$ definition of sequence convergence, $(b_{n})\to k$ if $a_{n}\leq b_{n} \leq c_{n}$, $(a_{n})\to k  $ and $(c_{n})\to k$. QED

\end{document}
