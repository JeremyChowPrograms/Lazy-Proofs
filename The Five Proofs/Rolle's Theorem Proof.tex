\documentclass[12pt]{article}
\usepackage{amsmath}
\usepackage{amssymb}
\usepackage{graphicx}
\usepackage{hyperref}
\usepackage[latin1]{inputenc}

\title{Rolle's Theorem Proof}
\author{Jeremy Chow}
\date{12/18/18}

\begin{document}
\maketitle
Rolle's theorem stated that if $f(x)$ in continuous on an interval $[a,b]$, differentiable on $(a,b)$ and $f(a)=f(b)=0$, the there exists at least one $c$ in $(a,b)$ where $f'(c)=0$. The proof can be separated into two cases with Case 1 being $f(x)=0$, then it's obvious that the statement holds true. Case 2 of the theorem can be solved by given the extreme value theorem, there exists an extrema at $x=c$ in $(a,b)$, which Fermat's theorem told us that $f'(c)=0$. QED
\end{document}