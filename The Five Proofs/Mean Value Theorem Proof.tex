\documentclass[12pt]{article}
\usepackage{amsmath}
\usepackage{amssymb}
\usepackage{graphicx}
\usepackage{hyperref}
\usepackage[latin1]{inputenc}

\title{Mean Value Theorem Proof}
\author{Jeremy Chow}
\date{12/18/18}

\begin{document}
\maketitle
Mean value theorem stated that if $f(x)$ in continuous on an interval $[a,b]$ and differentiable on $(a,b)$, the there exists at least one $c$ in $(a,b)$ where $f'(c)=\frac{f(b)-f(a)}{b-a}$. Let $g(x)=f(x)-\big(\frac{f(b)-f(a)}{b-a}(x-a)+f(a)\big)$, notice that $g(a)=g(b)=0$. By using Rolle's theorem, there exists a value $c \in (a,b)$ where $g'(c)=0$. Also notice that $g'(x)=f'(x)-\frac{f(b)-f(a)}{b-a}$. Combining both statements, we can get the equation\\$0=g'(c)=f'(c)-\frac{f(b)-f(a)}{b-a}\\0=f'(c)-\frac{f(b)-f(a)}{b-a}\\\frac{f(b)-f(a)}{b-a}=f'(c)\\f'(c)=\frac{f(b)-f(a)}{b-a}$\\QED
\end{document}