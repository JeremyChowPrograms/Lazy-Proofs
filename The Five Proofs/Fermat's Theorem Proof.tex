\documentclass[12pt]{article}
\usepackage{amsmath}
\usepackage{amssymb}
\usepackage{graphicx}
\usepackage{hyperref}
\usepackage[latin1]{inputenc}

\title{Mean Value Theorem Proof}
\author{Jeremy Chow}
\date{12/18/18}

\begin{document}
\maketitle
Fermat's theorem stated that if $f(c)$ is a local extrema of a differentiable function $f(x)$, then $f'(c)=0$. WLOG, assume $f(c)$ is the local maximum, then for $\delta$ being positive value approaching $0$, $\lim\limits_{\delta\to0^+}f(c+\delta)\leq\lim\limits_{\delta\to0^+}f(c)$, $\lim\limits_{\delta\to0^+}(f(c+\delta)-f(c))\leq0\\\lim\limits_{\delta\to0^+}\frac{f(c+\delta)-f(c)}{\delta}\leq0\\f'(c)\leq0$\\For $\delta$ being negative value approaching $0$, let $\delta'$ be absolute value of $\delta$, so $\lim\limits_{\delta'\to0^+}f(c-\delta')\leq\lim\limits_{\delta'\to0^+}f(c)\\\lim\limits_{\delta'\to0^+}(f(c)-f(c-\delta'))\geq0\\\lim\limits_{\delta'\to0^+}\frac{f(c)-f(c-\delta')}{\delta'}\geq0\\f'(c)\geq0$\\For both statements to hold true, $0\leq f'(c)\leq0$, $f'(c)=0$. QED
\end{document}